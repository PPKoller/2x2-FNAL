\section{ProtoDUNE-ND detector physics studies}
\label{sec:detector-physics-studies}

Any stability concerns or advantages from a detector operations point of view?\\\\ 

Detector physics tests:
\begin{itemize}
\item Rate and track multiplicity (Callum)
\item Neutrino event pile-up (Callum)
\item Track reconstruction when crossing modules (???)
\todo{stole this from the LOI}  The module walls produce gaps in particle tracks traversing multiple modules similar to dead wires in classic LArTPC readouts.
Algorithms to join such segmented tracks already exist~\cite{pandora}.
However, a detailed study of the influence of module walls on reconstruction efficiency still needs to be performed.

\item Charge sharing of EM and hadronic showers between modules (???)
\item Test for space charge build-up in a high intensity beam (maybe??? I guess this requires a laser or some sort of CRT to use muons...)
\item Electron-photon separation? Identified as an goal for charged particle test beam in the LOI (???)
\item Neutron studies... e.g., how well does ArcLight tag neutron scatters in different modules? (Patrick)
\item LAr --> Tracking detector matching efficiency (tracking detector input?). I guess we could keep this general, and keep open the possibility of using MINOS-ND or MINERvA if no other subdetectors can go in the hall for the test
\item Anything else???
\end{itemize}
