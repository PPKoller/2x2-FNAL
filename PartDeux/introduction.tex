\section{Introduction}
\label{sec:introduction}

The deployment of the ArgonCube 2x2 Demonstrator in the MINOS-ND hall as the core component of the of a prototype for the DUNE near detector (ND) was presented in the ProtoDUNE-ND proposal. The 2x2 would serve to as an engineering test stand for developing the techniques required ot deploy a full scale version of ArgonCube in the DUNE ND. This would also provide and opportunity to characterise the novel ArgonCube readout technologies response, and develop the reconstructions tools necessary for operating ArgonCube in the DUNE ND.      

The proposal noted that some fraction of the events seen in the ArgonCube 2x2 Demonstrator module would be fully contained except for the outgoing muon, which may open the way for extending the characterisation of detector response beyond the detector physics described in the proposal. 
Any possible extension in this direction would require a downstream tracker to contain higher momentum hadronic components, and to tag the muon. 
This makes the case for additional downstream ProtoDUNE-ND demonstrators for the HPTPC and 3DST.

In the absence of prototypes for the HPTPC or 3DST on a comparable scale as the 2x2, on the timeline of ProtoDUNE-ND, we propose repurposing component of existing detector to form a functionally identical detector array, and in doing so enhance the detector physics reach of ProtoDUNE-ND.      
By pairing the 2x2 with MINOS the response of 2x2 can be characterised as a function of momentum, providing a validation of the momentum measurement via multiple coulomb scattering that will need to be used for side-exiting muons in both the DUNE ND and far detector.
Pairing the 2x2 with MINERvA enables a test of the ability to track muons and fast neutrons from the slow and busy LAr detector into a fast-response detector, the keystone of the DUNE ND.

In this document, we expand on and present workable solutions for the potential detector physics enhancements to introduced in the ProtoDUNE-ND proposal. Studies of paring the 2x2 with components of MINERvA are presented in Section~\ref{sec:MINERvA}, including a discussion on track matching between fast and slow detectors and studies of acceptance and neutron tagging. The use of MINOS to provide a calibration of 2x2 as function of momentum is discussed in Section~\ref{sec:MINOS}. A cost estimate for the operation of the MINERvA components and method of reducing the running cost of MINOS is presented in Section~\ref{sec:costs}. Conclusions are presented in Section~\ref{sec:conclusions}.

