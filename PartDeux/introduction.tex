\section{Introduction}
\label{sec:introduction}

The deployment of the ArgonCube 2x2 Demonstrator in the MINOS-ND hall as the core component of ProtoDUNE-ND, a neutrino engineering testbench for the DUNE near detector (ND) has been presented in Ref.~\addcite. The ArgonCube 2x2 will serve to as an engineering test stand for developing the techniques required to deploy a full-scale version of ArgonCube in the DUNE ND. This provides an essential opportunity to characterize the novel ArgonCube readout technologies response, and develop the reconstruction tools necessary for operating ArgonCube in the DUNE ND.

Ref.~\addcite notes that although the detector physics case for the ArgonCube 2x2 alone is compelling, a large fraction of the high energy on-axis NuMI medium energy (ME) beam would not be fully contained in the ArgonCube 2x2 Demonstrator alone, and a number of further, or extended, detector physics studies would be possible if and when additional subdetector prototypes are introduced to ProtoDUNE-ND. The additional subdetector components considered are a Cosmic Ray Tagger (CRT) surrounding the ArgonCube component, and two downstream tracking detectors to contain higher momentum hadronic components, and to tag the muon. The two tracking detectors considered for DUNE ND are the high pressure argon gas TPC (HPgTPC) which will operate in a magnetic field to tag both the sign and momentum of charged particles, and the fully-active Three-Dimensional Scintillator Tracking detector (3DST), which will tag neutral and charged particles.

In the absence of prototypes for the HPgTPC or 3DST on a comparable scale as the 2x2, on the timeline of initial ProtoDUNE-ND operations in October 2020, in this document, we propose the ProtoDUNE-ND-Tracker, which uses repurposed MINERvA and MINOS-ND hardware, already on-axis in the NuMI medium energy (ME) beam, to provide a low cost tracking detector to enhance the detector physics reach of ProtoDUNE-ND.

%Pairing the 2x2 with MINERvA enables a test of the ability to track muons and fast neutrons from the slow and busy LAr detector into a fast-response detector, the keystone of the DUNE ND.
%By pairing the 2x2 with MINOS the response of 2x2 can be characterised as a function of momentum, providing a validation of the momentum measurement via multiple coulomb scattering that will need to be used for side-exiting muons in both the DUNE ND and far detector.

Studies of paring the 2x2 with components of MINERvA are presented in Section~\ref{sec:MINERvA}, including a discussion on track matching between fast and slow detectors and studies of acceptance and neutron tagging. The use of MINOS to provide a calibration of 2x2 as function of momentum is discussed in Section~\ref{sec:MINOS}. A cost estimate for the operation of the MINERvA components and method of reducing the running cost of MINOS is presented in Section~\ref{sec:costs}. Conclusions are presented in Section~\ref{sec:conclusions}.

%As all potential DUNE ND configurations identified in Ref.~\cite{dune_ndcsg} include a fast scintillator detector component, and a 3DST module will not be available on the timescale of ProtoDUNE-ND, it is desirable to see whether elements of the existing MINERvA detector in the NuMI hall could be re-purposed, as the MINERvA experiment will stop taking data in summer 2019, before ProtoDUNE-ND will get underway. This possibility is discussed separately in Section~\ref{sec:MINERvA_MINOS}.
