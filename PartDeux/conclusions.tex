\section{Conclusions}
\label{sec:conclusions}

In this document, we have outlined a solution for incorporating a downstream tracker into ProtoDUNE-ND, in order to expand the the detector physics reach beyond a stand alone ArgonCube 2x2 Demonstrator. 
This would improve containment and provide an essential test of combined reconstruction across different detectors with very different readout technology and times. 
All DUNE-ND designs considered in Ref.~\cite{dune_ndcsg} include some fast scintillator detector, but no prototype is foreseen at this time for ProtoDUNE-ND. 
As an illustrative example, we have considered incorporating elements of the soon to be decommissioned MINERvA detector into the ProtoDUNE-ND setup, provided all infrastructure and resource availability needs can be met. 
Additionally, we briefly discussed the possibility to include the MINOS-ND into ProtoDUNE-ND, which is due to be decommissioned along with the MINERvA experiment in mid-2019. In combination, using elements of MINERvA and MINOS-ND in ProtoDUNE-ND would allow us to contain all particles in a large fraction of events, and provide a realistic test of the energy reconstruction capabilities of the final DUNE-ND. This would be invaluable for DUNE-ND design studies, and would enhance the impact of the ProtoDUNE-ND test.
