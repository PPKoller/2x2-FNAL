\section{ProtoDUNE-ND configurations}
\label{sec:configurations}


It is highly desirable that the MINOS-ND will be removed from the hall, so that ProtoDUNE-ND can take advantage of the waterproof roof installed above it, although that decision has yet to be made. Here we outline two scenarios for re-using some components from the current MINERvA detector, which depend on whether the MINOS-ND can be removed.

In any scenario, we require 12 tracking modules from MINERvA to be places upstream and down stream of 2x2 for electric field calibration. The 2x2 should be aligned on-axis with the MINERvA modules to maximise the coverage for tagged crossing muons. 

\subsection{MINOS-ND removed}

In addition to the 12 downstream tracker modules, an ECal and HCal are required. The full 10 module ECal will provide electron/photon separation for particles produced in the 2x2. Only 2 to 3 modules of the HCal are needed to determine whether showers leaking out of the ECal are electromagnetic or not, aiding in photon/electron identification.

\subsection{MINOS-ND incorporated}

The 2x2 should be aligned such that acceptance is maximal at $\theta=0$, i.e. the 2x2 is on-axis with both MINERvA and MINOS-ND. This will ensure maximum momentum sensitivity for a muon originating in 2x2, regardless of the use of the magnet. 

If MINOS-ND is not removed, those parts from MINERvA which are not of interest can be removed in order from most upstream to most downstream. The remaining downstream portion could be left in situ.

In this scenario, the same 12 tracking modules and full
ECal will be left downstream, additionally, as the HCal cannot be removed from the most down-stream end of the detector, it should be retained.
