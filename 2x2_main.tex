\documentclass[aps,prd,preprint,superscriptaddress,nofootinbib]{revtex4-1}
\pdfoutput=1
\linespread{1}
\usepackage[T1]{fontenc}
\usepackage{amssymb}
\usepackage{url}
\usepackage{graphicx}
\usepackage{xspace}
\usepackage[svgnames]{xcolor}
\usepackage[caption=false]{subfig}
\usepackage{setspace}
\usepackage{amsmath}
\usepackage{listings} 
\usepackage[normalem]{ulem}
\usepackage{placeins}
%\usepackage{braket}
\usepackage{multirow}
\usepackage{tabulary}
\usepackage{hyperref}% add hypertext capabilities
\usepackage{dcolumn}% Align table columns on decimal point
\usepackage{bm}% bold math
\usepackage{soul}
%\usepackage{makecell}
\usepackage{siunitx}
\sisetup{separate-uncertainty=true}
\DeclareSIUnit\radlen{\text{\ensuremath{X_{\mathrm{0}}}}}
\DeclareSIUnit\clight{\text{\ensuremath{c}}} % remove 0 subscript from speed of light

\newcommand{\todo}[1]{{\color{red} (To Do: #1)}}
\newcommand{\blue}[1]{\textcolor{NALblue}{#1}}
\newcommand\addcite{[{\color{blue} \underline{CITATION NEEDED}}]}

\newcommand{\enu}{\ensuremath{E_{\nu}}\xspace}
\def\bracketbar{\hbox{\kern-9pt\raise1pt%
    \hbox{{\tiny(}{\lower1.5pt\hbox{\bf--}}{\tiny)}}}}

\begin{document}

\title{ProtoDUNE-ND: proposal to place the ArgonCube 2x2 Demonstrator module in a neutrino beamline at Fermilab}

\author{ArgonCube Collaboration}

\date{\today}

\begin{abstract}
  This document outlines the case for a neutrino beam test of the DUNE near detector (ND) at Fermilab, dubbed ProtoDUNE-ND. The proposed location for ProtoDUNE-ND is in the MINOS-ND hall, upstream of the current MINERvA/MINOS-ND locations, placing it on-axis in the NuMI medium energy neutrino beam. The core of ProtoDUNE-ND is the ArgonCube 2x2 Demonstrator module, a large scale demonstrator for the core liquid argon (LAr) component of the ND. A description of the ArgonCube 2x2 Demonstrator is given, together with a brief outline of the required infrastructure. Additionally, the possibility to include test modules for other DUNE ND component detectors is discussed. 
  An overview of the detector physics studies which can be completed with ProtoDUNE-ND is given, which will inform the final DUNE ND design choices, and aid in developing reconstruction tools in preparation for DUNE.
  Placing the 2x2 Demonstrator as a stand-alone underground in the NuMI beam will allow for the development and demonstration of the tools necessary to disentangle piled up events within the LAr.
  In the absence of DUNE ND demonstrators for the other technologies, existing detectors can be utilised.
  Pairing the 2x2 with MINERvA enables a test of the ability to track muons and fast neutrons from the slow and busy LAr detector into a fast-response detector, the keystone of the DUNE ND. 
  Additionally, by pairing the 2x2 with both MINERvA and MINOS the response of 2x2 can be characterised as a function of momentum, providing a validation of the momentum measurement via multiple coulomb scattering that will need to be used for side-exiting muons in both the DUNE ND and far detector.
  
\end{abstract}

\maketitle

% For now suppress the table of contents
%\tableofcontents

\section{Introduction}
\label{sec:introduction}

DUNE is a highly ambitious next-generation accelerator neutrino oscillation experiment~\addcite. DUNE will utilise an intense beam of predominantely muon (anti)neutrinos produced at Fermilab, sample the unoscillated beam on the Fermilab site, ~650m \todo{check!} from the proton target, and then sample the unoscillated beam XXkm away at XXX, using 4 10kt liquid argon (LAr) target modules. By using LAr time projection chambers (TPCs), DUNE benefits from the exquisite precision to look at the particles escaping the nucleus, and measuring the neutrino energies with unparalleled precision. \todo{Read a DUNE intro paper and refine this to toe the party line.} An extensive research and development (R\&D) program is currently ongoing to deliver detectors capable of fulfilling DUNE's physics goals. The unprecedented size of the DUNE far detectors has motivated large-scale LAr TPC tests, in the single- and dual-phase ProtoDUNE modules in CERN, in themselves, the largest LAr TPCs built to date. Similarly, the intense neutrino flux at the near detector has motivated a program of R\&D into pixelized charge readout, and various improvements with respect to traditional LAr TPCs, in the ArgonCube program \todo{make this less shit, and say modularized}. A necessary final step in this program is a full size test of the ArgonCube modules \todo{mention that ProtoDUNE are conceived as full-size modules of the FD} in a neutrino beam to check that it can perform as required by DUNE, and to guide reconstruction development work to be ready for the full DUNE near detector when it is built in XXXX.

Introduction to DUNE and the DUNE ND. Describe and reference past R\&D work done for ArgonCube, ArgonTube, etc. \\ \\

Add similar descriptions for HPTPC and 3DST if they will be able to form a part of this proposal. If not, I guess we should comment on the general multi-detector ND design, but not go into details. {\it If we ultimately do not include anything other than the 2x2, we should comment on the location of MINERvA/MINOS-ND in the hall, and the possibility to test reconstruction with multiple detectors... maybe that's a useful think to also test?}\\\\

Proposal on the table is to put test modules into the NuMI on-axis neutrino beamline at FNAL to act as ProtoDUNE-ND --- vital test of the near detectors before investing in the full DUNE ND. A description of the proposal location and test modules to be used in Section~\ref{sec:protodune-nd}. A description of the detector physics studies to be performed in Section~\ref{detector-physics-studies}. The outlook for the test modules and a timeline in Section~\ref{sec:outlook}. {\it Probably want to have a strong show of support from FNAL in this section --- e.g. if they promose concrete infrastructure support, it should go here, as well as in the description of the location for the test.} Conclusions in Section~\ref{sec:conclusions}.



\section{ProtoDUNE-ND}
\label{sec:protodune-nd}

This section outlines the proposed ProtoDUNE-ND demonstrator modules for various components of the 

\section{ProtoDUNE-ND detector physics studies}
\label{sec:detector-physics-studies}
Basic detector stability checks will be performed with a period of detector operation in Bern before moving the ArgonCube 2x2 Demonstrator module to Fermilab. These tests will include extraction and re-insertion tests of individual modules into the LAr bath, and checks that the LAr purity is sufficient. However, local tests can only be performed using cosmic muons, which have limited utility beyond basic detector stability checks. In this section, we identify a number of key detector physics questions which could be answered by the ProtoDUNE-ND test, and would help inform the final design of the DUNE ND, and aid in developing reconstruction algorithms suitable for neutrino interactions.

In order to check the feasibility of these studies, two different simulations were used. Firstly, high statistics GENIE Monte Carlo samples were produced, in order to compare basic properties of neutrino interactions expected in the LBNF and NuMI ME beamlines. Secondly, GENIE events were used to seed a basic GEANT4 simulation, using the ArgonBox\footnote{\url{https://github.com/dadwyer/argon_box}} software, in order to get a basic understanding of event shape and containment. In the latter simulation, events were simulated in a very large (200 m $\times$ 200 m $\times$ 200 m) box of LAr, and were then distributed randomly inside a volume with the correct spatial dimensions as the 2x2 Demonstrator module. Although the 2x2 geometry was not included in the simulation, this gives an acceptable estimate of the expected event rates for the studies described below. Examples of the ArgonBox simulation with the basic 2x2 Demonstrator geometry superimposed can be seen in Figure~\ref{fig:argonbox_event_display} for a number of different neutrino energies.

\begin{figure}[htb]
  \centering
  \subfloat[$E_{\nu}$ = 2.60 GeV] {\includegraphics[width=0.45\textwidth]{{plots/EventDisplays/2.60GeV_square_crop}.png}}\hspace{25pt}
  \subfloat[$E_{\nu}$ = 3.36 GeV] {\includegraphics[width=0.45\textwidth]{{plots/EventDisplays/3.36GeV_square_crop}.png}}\\
  \subfloat[$E_{\nu}$ = 4.83 GeV] {\includegraphics[width=0.45\textwidth]{{plots/EventDisplays/4.83GeV_square_crop}.png}}\hspace{25pt}
  \subfloat[$E_{\nu}$ = 9.37 GeV] {\includegraphics[width=0.45\textwidth]{{plots/EventDisplays/9.37GeV_square_crop}.png}}
  \caption{Example $\nu_{\mu}$--argon ArgonBox simulated events for a number for different incident neutrino energies, where the energy deposits in a bulk volume of LAr are color-coded according to the particle type: $\pi^{\pm}$ --- blue; $\mu^{\pm}$ --- purple; $e^{+}$ --- green; $e^{-}$ --- yellow; proton --- red; recoiling nuclei --- black. The event vertices are randomly placed within the active volume of the 2x2 Demonstrator module, the geometry for which is superimposed on these images, but which is not simulated by ArgonBox.}
  \label{fig:argonbox_event_display}
\end{figure}

The example event displays shown in Figure~\ref{fig:argonbox_event_display} give a basic idea of how NuMI ME energy events (in FHC) would look in the ArgonCube 2x2 Demonstrator module. Although many of the tracks and showers are not contained, some fraction are, which is discussed in more detail for the detector physics studies described below. It also suggests that some fraction of the events seen in the 2x2 Demonstrator module would be fully contained in the active volume, or fully contained except for the outgoing muon, which may open the way for some interesting physics studies beyond the detector physics described in this document. However, any possible utility in this direction would be severly limited without a downstream tracker to contain higher momentum hadronic components, and to tag the muon. This makes the case for adding an additional ProtoDUNE-ND test module for a HPTPC downstream tracker even stronger, or if that proves not to be possible on the timescale of this test, utilizing the existing MINERvA or MINOS-ND detectors in the experimental hall.

\begin{figure}[htb]
  \centering
  \includegraphics[width=0.5\textwidth]{plots/2x2_ntracks_all.png}
  \caption{The expected yearly rates of minimum and highly ionizing particles expected in the 2x2 Demonstrator module's 1.7t LAr volume for the NuMI ME and LBNF fluxes, produced using GENIE v2.12.10 with the ``ValenciaQEBergerSehgalCOHRES'' configuration~\cite{genie}.}
  \label{fig:track_multiplicity}
\end{figure}
In order to be a relevant test for the full ArgonCube near detector, which will be in the LBNF beamline, it is useful to verify that the basic properties of the events are similar, despite the NuMI ME beam being somewhat higher energy than the planned LBNF beam (as shown in Figure~\ref{fig:beam_options}). Figure~\ref{fig:track_multiplicity} shows the expected multiplicity of minimum or highly ionizing tracks at the vertex for both the LBNF and NuMI ME beams, in neutrino and antineutrino mode, produced with the GENIE generator. The track multiplicities are similar, which indicates that the scale of the reconstruction problem is similar, and the proposed ProtoDUNE-ND test will be a useful benchmark for developing the ArgonCube reconstruction software.

\begin{figure}[htb]
  \centering
  \subfloat[$\mu^{\pm}$] {\includegraphics[width=0.5\textwidth]{plots/2x2_muon_mom_all.png}}
  \subfloat[Protons]    {\includegraphics[width=0.5\textwidth]{plots/2x2_proton_mom_all.png}}\\
  \subfloat[$\pi^{+}$]   {\includegraphics[width=0.5\textwidth]{plots/2x2_piplus_mom_all.png}}
  \subfloat[$\pi^{-}$]   {\includegraphics[width=0.5\textwidth]{plots/2x2_piminus_mom_all.png}}  
  \caption{The expected yearly rates of various particles produced at the vertex, as a function of their momentum, expected in the 2x2 Demonstrator module's 1.7t LAr volume for the NuMI ME and LBNF fluxes, produced using GENIE v2.12.10 with the ``ValenciaQEBergerSehgalCOHRES'' configuration~\cite{genie}. Note that every relevant particle from each event is included.}
  \label{fig:momenta}
\end{figure}
In Figure~\ref{fig:momenta}, the momenta of various particles coming from the initial neutrino--argon vertex are compared for the LBNF and NuMI ME beams. As expected, the energy distributions of all of the particles are slightly broader for the NuMI ME flux, but there are significant numbers of events with particle kinematics across the broad range of energies expected for the LBNF beams.

In the full 7 $\times$ 5 module ArgonCube detector and the more intense LBNF beamline, pile-up will be a further challenge, with $\sim$14.7 interactions per 10 $\mu$s beam spill. With a drift window of 220 $\mu$s, the pixel readout will not be able to desentangle overlapping events. Here, the ArCLight light-readout system, with the ability to measure prompt scintillation light with nanosecond resolution, will play a crutial role to associate particle tracks with the correct interaction vertices. Additionally, the relatively small size of the 2x2 Demonstrator module means that relatively few of the tracks will be contained, making particle identification (PID) studies challenging, except for the cases listed below. Although other detectors are not included in the ArgonBox simulation, the lack of containment and PID capabilities mean that including another subdetector in the ProtoDUNE-ND setup is essential for any ancilliary physics measurements to be made.

\FloatBarrier
\subsection{Combining light and charge signals}
An important challenge is to develop automated event reconstruction software with the ArgonCube detector. The pixel readout removes the ambiguities present for projective wire readout TPCs, but the reconstruction software for the latter has benefitted from several years of development for the MicroBooNE~\cite{microboone} and ICARUS experiments~\cite{icarus}. Although strides forward for pixel readout TPCs have been made in the PixLAr experiment (where pixel planes were introduced to the LArIAT experiment~\cite{lariat}), the reconstruction problem for charge particle scattering in a small TPC is much simpler than for the ProtoDUNE-ND or DUNE ND environments. Additionally, the recomstructed track position along the drift direction, and the suppression of cosmic backgrounds within the beam window, will be performed using information from the ArcLight light collection system. Checking that the light and charge signals can be combined in the full-size ArgonCube modules, in a comparably noisy environment to the DUNE ND, is an essential test of the ArgonCube design.

\subsection{Neutron identification}
Neutrons present a particular challenge for neutrino energy reconstruction in DUNE and other long-baseline neutrino oscillation experiments. Neutrino oscillations are a function of neutrino energy, but this cannot be reconstructed on an event by event basis because neutrons carry away some fraction of the energy, and are not directly observable. Figure~\ref{fig:neutron_kinematics} shows the expected neutron rate as a function of multiplicity and momentum for the LBNF and NuMI ME beamlines.
\begin{figure}[htb]
  \centering
  \subfloat[Neutron multiplicity]   {\includegraphics[width=0.5\textwidth]{plots/2x2_nneutron_all.png}}
  \subfloat[Neutron momentum]       {\includegraphics[width=0.5\textwidth]{plots/2x2_neutron_mom_all.png}}  
  \caption{The expected yearly rates of neutrons produced at the vertex, as a function of event multiplicity and their momentum, expected in the 2x2 Demonstrator module's 1.7t LAr volume for the NuMI ME and LBNF fluxes, produced using GENIE v2.12.10 with the ``ValenciaQEBergerSehgalCOHRES'' configuration~\cite{genie}. Note that every neutron from each event is included in the momentum distribution.}
  \label{fig:neutron_kinematics}
\end{figure}

There are two possible ways to identify the presence of neutrons in an event:
\begin{enumerate}
\item Detection of slow neutrons with kinetic energies of $\mathcal{O}$~keV through neutron capture. This is usually achieved by applying a neutron affine coating, e.g. gadolinium, to the detector walls. The radiation emitted from excited nuclei after having captured a neutron is interaction-specific and thereby serves as an indicator for slow neutrons.
\item Detection of fast neutrons with kinetic energies from $\mathcal{O}$~1 MeV-- 1 GeV through recoiling charge particles after a collision of a neutron with a nucleus. The recoiling particle can be the nucleus as a whole, or, if the neutron exceeds the nuclear binding energy ($\sim$~5~MeV for an argon nucleus), a knock-out proton or heavier nuclear fragments like deuterons.
\end{enumerate}
For oscillation experiments, fast neutrons are a much greater issue, as they may carry away a significant fraction of the neutrino energy in an event. It is, therefore, of great interest to investigate the potential of LAr experiments to tag these missing neutron with neutron-induced recoils. Investigating the neutron tagging rate in ProtoDUNE-ND will provide useful information for DUNE sensitivity studies as it will in the ProtoDUNE-ND, this provides an opportunity to investigate how well charge and light signals can be combined.

\begin{figure}[htbp]
  \centering
  \includegraphics[width=0.8\textwidth]{plots/primary_neutron_recoils.png}
  \caption{Kinetic-energy distribution of secondary particles with respect to incident neutron kinetic energy for neutron interactions in LAr, shown for 100,000 simulated neutrino events (which may have more than one neutron produced at the vertes).}
  \label{fig:neutron_recoils}
\end{figure}
Figure~\ref{fig:neutron_recoils} shows the kinetic energy of secondary particles after the interaction of a primary neutron in LAr. The horizontal red line gives the energy threshold for the detection of these secondary particles with ArCLight~\cite{arclight}. Clearly, both recoiling argon nuclei and recoiling protons have kinetic energies well above the ArCLight detection threshold. While recoiling argon nuclei show typical energies between 100~keV and 1~MeV, recoiling protons show energies $>$~1~MeV, up to several GeVs.

\begin{figure}[htbp]
  \centering
  \includegraphics[width=0.8\textwidth]{plots/proton_track_length.png}
  \caption{Track length of recoiling protons for neutrons produced in 100,000 neutrino interactions. About 30\% of all recoils are resolvable as tracks with the LArPix pixel charge-readout system.}
  \label{fig:proton_length}
\end{figure}
Given the LArPix $\sim$~3~mm pixel-pitch (see Section~\ref{sec:2x2-design}), the minimum reconstructable track length in ArgonCube is also $\gtrsim$~3~mm. Figure~\ref{fig:proton_length} shows the track length of recoiling protons with respect to the primary neutron kinetic energy. Recoiling protons can, depending on their energy, produce tracks which are up to $\sim$10~cm long. About 30~\% of all recoiling protons are resolvable by the pixelated charge readout, which correspond to protons that are knocked out of a nucleus by primary neutrons with energies $\gtrsim$~50~MeV.

\begin{figure}[htbp]
  \centering
  \includegraphics[width=0.8\textwidth]{plots/min_dist_vtx_proton_recoil.png}
  \caption{The minimum distance between the neutrino vertex and the neutron-induced proton track, as a function of neutron kinetic energy. Produced with 100,000 initial neutrino events simulated by ArgonBox.}
  \label{fig:min_dist_proton}
\end{figure}
Figure~\ref{fig:min_dist_proton} shows the minimum distance between the neutrino vertex and the neutron-induced proton track, as a function of neutron kinetic energy. The majority of proton recoils occur within 1 m, so many neutron-induced proton recoils will be contained within the 2x2 Demonstrator module.

\FloatBarrier
\subsection{Reconstruction in a modular environment}
\todo{James, please clarify and expand this!}
The module walls of the ArgonCube design produce gaps in particle tracks traversing multiple modules similar to dead wires in classic LArTPC readouts. Algorithms to join such segmented tracks already exist~\cite{pandora}, but have not been adapted to the ArgonCube design. Simple track matching efficiencies across modules can be calculated using cosmics, which will be an essential first step. However, for events with many tracks produced at the vertex (see Figure~\ref{fig:track_multiplicity}), a detailed study of the reconstruction performance given the module walls will need to be carried out. ProtoDUNE-ND provides an opportunity to do so, and to check that the reconstruction behaves as expected, before moving to the full DUNE ND deployment of ArgonCube.

\begin{figure}[htb]
  \centering
  \subfloat[EM shower]       {\includegraphics[width=0.5\textwidth]{plots/2x2_minerva_plots/EM_cont_eff_2x2.png}}
  \subfloat[Hadronic shower] {\includegraphics[width=0.5\textwidth]{plots/2x2_minerva_plots/H_cont_eff_2x2.png}}
  \caption{Containment efficiency for EM and hadronic showers produced by an interaction within the ArgonCube 2x2 active volume, as a function of initiator particle energy and angle w.r.t the incoming beam direction. Note that if $\geq 90$\% of energy is deposited within the 2x2 active volume, it is classed as contained.}
  \label{fig:2x2_shower_containment}
\end{figure}
This problem becomes significantly more complicated for electro-magnetic (EM), or hadronic, showers which cross modules. ProtoDUNE-ND will provide an opportunity to develop reconstruction software, and check how well it performs for realistic shower energies for neutrino interactions which cover the neutrino energy range of interest for the LBNF beamline. At these energies, shower development is known to be problematic, and shower reconstruction in LAr is a significant challenge. Additionally, in order to test how well the reconstruction can identify shower depth, a sample of fully contained showers would be extremely useful. Figure~\ref{fig:2x2_shower_containment} shows the efficiency to fully contain EM-showers or hadronic showers produced by an interaction within the ArgonCube 2x2 active volume, as a function of initiator particle energy and angle w.r.t the incoming beam direction. Note that if $\geq 90$\% of energy is deposited within the 2x2 active volume, it is classed as contained. \todo{Add some event displays.}

\todo{Patrick, James, can you think of anything else here? Seems a bit weak on the shower study...}

\subsection{$\pi^{0}$ reconstruction}
A more quantitative measure of how well EM showers can be reconstructed in the modularized ArgonCube detector could be possible using $\pi^{0} \rightarrow \gamma\gamma$ decays (which have a branching ratio of 98.8\%~\cite{pdg_2018}), in which both decay photons produce a shower, and are contained in the active volume of the detector. Combining the information on the two showers, and attempting to reconstruct the invariant mass peak of the $\pi^{0}$ provides a measurement of the EM shower resolution.

\begin{figure}[htb]
  \centering
  \subfloat[$\pi^{0}$ multiplicity]   {\includegraphics[width=0.5\textwidth]{plots/2x2_npi0_all.png}}
  \subfloat[$\pi^{0}$ momentum] {\includegraphics[width=0.5\textwidth]{plots/2x2_pi0_mom_all.png}}  
  \caption{The expected yearly rates of $\pi^{0}$'s produced at the vertex, as a function of event multiplicity and their momentum, expected in the 2x2 Demonstrator module's 1.7t LAr volume for the NuMI ME and LBNF fluxes, produced using GENIE v2.12.10 with the ``ValenciaQEBergerSehgalCOHRES'' configuration~\cite{genie}. Note that every $\pi^{0}$ from each event is included in the momentum distribution.}
  \label{fig:pi0_kinematics}
\end{figure}
Figure~\ref{fig:pi0_kinematics} shows the expected $\pi^{0}$ production rate in the active volume of the 2x2 Demonstrator module in the LBNF NuMI ME beamlines, as a function of $\pi^{0}$ multiplicity in each event and $\pi^{0}$ momentum. There are, of course, some qualifiers for this study. Many photon-induced showers will not be contained, and those which are will be lower energy than many EM showers expected in the DUNE ND. Figure~\ref{fig:pi0_containment_2x2} shows the efficiency for containing both photon-induced showers from a primary $\pi^{0}$ decay in the ArgonCube 2x2's active volume, shown for all $\pi^{0}$'s produced inside that volume. As expected, the efficiency is low for high energy pions, but it will still be possible to reconstruct a large fraction of the lower momentum $\pi^{0}$'s from Figure~\ref{fig:pi0_kinematics}. Overall, this study shows that this $\pi^{0}$ mass peak reconstruction will be a worthwhile study at ProtoDUNE-ND.

\begin{figure}[htb]
  \centering
  \includegraphics[width=0.5\textwidth]{plots/2x2_minerva_plots/Pi0_cont_eff_2x2.png}
  \caption{Efficiency for containing both photon-induced showers from $\pi^{0}$ decays in the ArgonCube 2x2 module, as a function of the $\pi^{0}$ kinetic energy and angle w.r.t the incoming neutrino direction. Containment is defined as $\geq$90\% of the energy being deposited in an active volume of a detector, and all primary $\pi^{0}$'s produced inside the 2x2's active volume are included.}
  \label{fig:pi0_containment_2x2}
\end{figure}
Two further issues for this study are apparent. Firstly, events with more than one $\pi^{0}$ introduce a problem: even if two EM showers are fully contained, they may not come from the same $\pi^{0}$ decay. Secondly, of those $\pi^{0}$ decays for which both photons are fully contained, the initial $\pi^{0}$ is likely to have a low momentum, which is likely to exclude some fraction of the events. However, despite these challenges, a measure of EM shower resolution from ProtoDUNE-ND would be very useful for DUNE ND design studies, so this possibility is worth investigating further.
\todo{James, do you have any other comment on photon/electron separation? I decided this would be a better thing to aim for for Patrick's studies}
\FloatBarrier
%\subsection{Michel tagging}
%\todo{Figure out if we can do anything remotely reasonable... Patrick could see how many stopped muons/pions he has. But I'm not sure how we get the efficiency from that...}

%\subsection{Proton tagging}
%\todo{Worth commenting on figuring out how well we can find Bragg peaks? Do we care for this test?}

\subsection{Electric field uniformity and space charge build-up}
A concern for LAr detectors in a high intensity beam is the build up of space charge --- long-lived argon ions which drift slowly towards the cathode --- and possible affects on the uniformity of the electric field which may accumulate over time. In currently operating and near-future LAr detectors~\cite{Ereditato:2014lra, Antonello:2015lea}, both cosmic tracks and UV lasers are used to calibrate for distortions in the electric field. Both the UV laser track and high energy cosmic muons are expected to leave straight tracks in the detector. If the drift field is not uniform across the detector, ionization electrons produced along the length of this track will not drift at the same speed, and will result in a distorted track at the readout plane. By comparing the reconstructed and expected track, a map of the electric field distortion can be built up for calibration purposes.

Assuming that the 2x2 Demonstrator module is equipped with scintillator panels to tag cosmic tracks using a timing coincidence between two sides of the detector, and reasonable spatial resolution on those scintillator paddles, electric field distortions could be measured in the 2x2 Demonstrator module. By looking at beam-on, and beam-off data, it would be possible to look at the possible affect of space charge build-up over time due to the high event rate in the NuMI beam. If significant space charge build-up were observed, this would inform the future ArgonCube ND design, as a higher drift field strength would be required.

Additionally, although the electric-field uniformity of the resistive field shell will be checked in a small scale LAr TPC at Bern, a check of the electric-field uniformity and stability over time for full-size ArgonCube modules would be a valuable final validation of the design.

\subsection{Cosmic suppression}
\todo{James, check whether this makes sense}

The DUNE ND will be a surface detector, with a relatively small overburden. ArgonCube, and all LAr TPCs are slow detectors, with long drift times, over which multiple cosmics will leave tracks in the detector. Although fast timing from the light collection system can, in principle, strongly suppress the cosmic background, this is a reconstruction challenge. A key design choice for the DUNE ND is whether or not a cosmic ray tagging system is required --- for example, a series of scintillator planes surrounding the LAr TPC, as for SBND and MicroBooNE~\cite{CRT}. The proposed ProtoDUNE-ND test experiment can help inform this design choice, assuming that the ArgonCube 2x2 Demonstrator module is equipped with scintillator paddles.

By using the scintillator paddles to tag cosmic events using a timing coincidence, methods for rejecting cosmics can be validated independently. For example, we expect that the good timing resolution and light localization from ArcLight will allow cosmics out of the beam window to be rejected with a high efficiency. Additionally, if a cosmic muon traverses a pixel plane, or an ArcLight plane, we would expect to see a large charge deposition or a large number of photoelectrons measured, which could be an additional way to reject cosmic muons. Both of these methods can be validated with ProtoDUNE-ND.

\subsection{Reconstruction with multiple subdetectors}
Because the ArgonCube 2x2 Demonstrator module is relatively small, many events from the NuMI ME beam will not be contained. Figure~\ref{fig:leaky_event} shows a neutral current event where many pions are produced, but in which the pions and subsequent hadronic showers extend far beyond the detector. Such events are likely to be uncontained even in the full ArgonCube component of the DUNE ND. In charged-current events, the muon will be uncontained most of the time. For this reason, and because it is not possible to magnetize the large LAr component, a magnetized tracking detector is proposed downstream of ArgonCube in the DUNE ND. A test module for the HPTPC is hoped to be included as part of the ProtoDUNE-ND effort as described in Section~\ref{sec:tracking_detectors}. With multiple subdetectors included in ProtoDUNE-ND, multi-detector reconstruction capabilities can be developed and tested. Additionally, if the sign and momentum of escaping hadrons and muons can be measured, it may be possible to make physics measurements with ProtoDUNE-ND, which would be beneficial to the overall DUNE program. For that reason, if no downstream tracker is available, it would be highly desirable to utilize the proximity of the MINERvA and MINOS-ND detectors, and try to minimize the distance between the ArgonCube 2x2 Demonstrator module and those detectors, in order to use them as downstream trackers.
\begin{figure}[htb]
  \centering
  \includegraphics[width=0.8\textwidth]{{plots/EventDisplays/8.17GeV_rectangle_crop}.png}
  \caption{Example ArgonBox simulated event for an 8.17 GeV $\nu_{\mu}$--argon neutral-current multi-pion interaction, in which the pions are not contained in the module. Energy deposits in a bulk volume of LAr are color-coded according to the particle type: $\pi^{\pm}$ --- blue; $\mu^{\pm}$ --- purple; $e^{+}$ --- green; $e^{-}$ --- yellow; proton --- red; recoiling nuclei --- black. The event vertex was randomly placed inside the active volume of the 2x2 Demonstrator module, the geometry for which is superimposed on these images, but which is not simulated by ArgonBox.\todo{Can the full size ND be superimposed on this as well?}}
  \label{fig:leaky_event}
\end{figure}
\FloatBarrier

\section{Incorporating a downstream tracker into ProtoDUNE-ND}
\label{sec:MINERvA}

As discussed in Section~\ref{sec:detector-physics-studies}, a large fraction of particles will not be contained in the 2x2 alone. An additional tracking component would recover many of these particles, and would make it possible to benchmark the 2x2 response to a wider range of particle energies, and therefore the expected DUNE phase-space. One possibility raised by MINERvA collaborators is to repurpose portions of their detector after its decommissioning.      

This section discusses enhancements to the detector physics program at ProtoDUNE-ND if a downstream tracker is included, based on discussions with members of MINERvA and other active DUNE ND collaborators who have expressed an interest in joining the ProtoDUNE-ND effort. Although, these studies are not specific to MINERvA and MINOS ND, we have used them as illustrative examples here because MINERvA collaborators have offered advice and engineering support.   

The MINERvA experiment~\cite{minerva-nim}, which now controls and maintains the re-purposed MINOS near detector, is located in the MINOS ND hall in which ProtoDUNE-ND will be installed, and is due to complete its data-taking in spring--summer 2019. As discussed in Section~\ref{sec:protodune-nd}, all DUNE ND designs considered in Ref.~\cite{dune_ndcsg} include some fast scintillator component. Combining slow LAr TPC charge readout and fast scintillator and LAr light readout will be an essential reconstruction requirement for the DUNE ND. 

We note that clearly, the scientific benefit in having a downstream tracker included in ProtoDUNE-ND, or repurposing an existing tracker, must be evaluated against the technical feasibility and the availability of resources (person power and financial). These costs would have to be shared between ProtoDUNE-ND collaborators and Fermilab. 

\subsection{Repurposing the MINERvA detector}
\label{sec:minerva}
The MINERvA detector is shown in Figure~\ref{fig:minerva_detector}. For the purpose of the discussion in this section, we will assume that the steel shield, scintillator veto plane, helium vessel, and nuclear targets region will be removed. 
\begin{figure}[htb]
  \centering
  \subfloat[Front view] {\label{fig:minerva_detector_front}\includegraphics[width=0.35\textwidth]{plots/minerva_module_transverse}}
  \subfloat[Side view]  {\includegraphics[width=0.60\textwidth]{plots/MINERvA_schematic.pdf}}
  \caption{Schematic of the MINERvA experiment. Reproduced from Figure 1 of Ref.~\cite{minerva-nim}.}
  \label{fig:minerva_detector}
\end{figure}

The remaining central tracking region, and both the electromagnetic(EM) and hadronic calorimeters are divided into modules which consist mostly of hexagonal scintillator planes, each made of 127 triangular scintillator bars, arranged in three different orientations (60$^\circ$ rotations between each plane). There are 62 modules in the fully-active tracker region, each composed of two scintillator layers. A 15 cm border of 0.2 cm thick lead on the downstream end of each module provides EM calorimetry for particles exiting the side of the tracking region.

The downstream EM calorimeter is composed of 10 modules, each with two scintillator planes and a 0.2 cm thick lead plate on the downstream end. There are 20 modules in the downstream hadronic calorimeter, each with a single scintillator plane and a 2.54 cm thick hexagonal steel plane. The outer detector consists of a steel frame supporting structure with embedded scintillator planes, as can be seen in Figure~\ref{fig:minerva_detector_front}, which turns the support structure into a hadronic calorimeter. The combination of the downstream and side EM and hadronic calorimeters allows for containment of most particles which escape from the central tracking region, and allows for particle identification and momentum measurements.

\begin{figure}[htb]
  \centering
  \includegraphics[width=0.8\textwidth]{{plots/Event_Displays_2x2_MINERvA/MINERvA_full_e70_rectangle_crop}.png}
  \caption{Example simulated event for a 7.0 GeV $\nu_{\mu}$--argon charged-current interaction, in which particles not contained in the ArgonCube 2x2 enter the MINERvA central tracking region downstream. Energy deposits are color-coded according to the particle type: $\pi^{\pm}$ --- blue; $\mu^{\pm}$ --- purple; $e^{+}$ --- green; $e^{-}$ --- yellow; proton --- red; recoiling nuclei --- black. The event vertex was randomly placed inside the active volume of the 2x2 Demonstrator module.}
  \label{fig:2x2+MINERvA_event}
\end{figure}

For the studies shown in this section, a simulation was performed approximating the downstream MINERvA central tracking region with a box of scintillator, and the ArgonCube 2x2 Demonstrator module upstream of the shortened MINERvA detector. Neutrino interactions are generated in the ArgonCube active volume, and propagated through an approximation of the ArgonCube 2x2 demonstrator and partial MINERvA detector with a Geant4-based program.  The MINERvA detector is approximated by a rectangular box, \SI[product-units=repeat]{1.4x1.4}{\metre\squared} in the dimensions transverse to the beam.  The simulation includes the most downstream 12 modules (24 planes) of the tracker region, as well as the full downstream ECAL and HCAL regions.  The rectangular box represents the central part of the MINERvA inner detector, and is large enough to cover the entire ArgonCube active volume. An example event is shown in Figure~\ref{fig:2x2+MINERvA_event}, and can be compared with 2x2 only events in Figures~\ref{fig:argonbox_event_display} and~\ref{fig:leaky_event}. Note that this simulation only included the ArgonCube cryostat and MINERvA detector components, no material was included outside these (so escaping particles simply leave without ever re-interacting). This is unlike the previously described ArgonBox simulation, where a large box of argon was simulated (so escaping particles still re-interact). Events were again distributed uniformly throughout the ArgonCube active volume.

In the following, we discuss potential detector physics studies, or improvements to detector physics studies previously discussed in the ArgonCube 2x2-only case (in Section~\ref{sec:detector-physics-studies}), incorporating elements of the MINERvA detector downstream of the ArgonCube 2x2 module.

\subsubsection{Track matching}
All DUNE ND designs considered in Ref.~\cite{dune_ndcsg} include some fast scintillator component, downstream of the LAr ArgonCube component, and downstream of a low-density GAr TPC tracker, to tag escaping particles, photons, and possibly neutrons. There is a significant reconstruction challenge in matching the escaping tracks from the LAr component, with the signals in the scintillator, given the slow charge readout in the LAr TPC, and the high multiplicity DUNE-ND environment.

\begin{figure}[htb]
  \centering
  \includegraphics[width=0.6\textwidth]{plots/2x2_minerva_plots/track_mathch_multiplicity.png}
  \caption{Simulated number of true tracks produced by simulated GENIE interactions in the ArgonCube 2x2 active volume, which deposit energy in both the 2x2 module, and the MINERvA component positioned downstream of the 2x2.}
  \label{fig:track_multiplicity_min}
\end{figure}
Many tracks produced in the LAr volume are not contained by the ArgonCube 2x2 module, and the majority will escape downstream. In Figure~\ref{fig:hadronic_containment}, the multiplicity of tracks which deposit charge in both the ArgonCube 2x2 module and the MINERvA component, included in the simulation described above, are shown. Full DUNE-ND events are likely to have an even higher LAr to scintillator track multiplicity due to the pile-up in the much larger 35 t ArgonCube LAr detector. But it is clear from Figure~\ref{fig:track_multiplicity_min} that including MINERvA elements in the ProtoDUNE-ND tests would provide useful data with which to start tackling this reconstruction problem.

\begin{figure}[htb]
  \centering
  \includegraphics[width=0.6\textwidth]{plots/2x2_minerva_plots/track_mathch_topo.png}
  \caption{Simulated number of true tracks produced by simulated GENIE interactions in the ArgonCube 2x2 active volume, which exit the downstream face of the 2x2 module, relative to the number of tracks which enter the upstream face of the downstream MINERvA component, event by event.}
  \label{fig:track_multiplicity_topo}
\end{figure}
As can be seen from the event display shown in Figure~\ref{fig:2x2+MINERvA_event}, events in which tracks escaping the 2x2 active volume may re-interact in the surrounding LAr bath before entering the MINERvA component included in this simulation, thus making the event more confusing, and difficult to assess reconstruction performance with. Figure~\ref{fig:track_multiplicity_topo} shows the multiplicity of tracks exiting the downstream face of the 2x2 active volume downstream, compared with the number of tracks entering the upstream face of the MINERvA component included in the simulation. The distribution is fairly diagonal, suggesting that although complicated event topologies exist, the events will not be too confused to use for these studies. Note also that this problem could be dramatically reduced by partially instrumenting the dead region between the two detectors.

\subsubsection{Acceptance studies}
\label{sec:minerva-acceptance}
The inclusion of MINERvA in ProtoDUNE-ND will improve the acceptance of particles for various studies. Here, we show how the efficiency for contained events compares for the 2x2+MINERvA setup desribed above, with MINERvA components located downstream of the ArgonCube 2x2 Demonstrator module, and for the 2x2-only case.

\begin{figure}[htb]
  \centering
  \subfloat[2x2-only]    {\includegraphics[width=0.45\textwidth]{plots/2x2_minerva_plots/H_cont_eff_2x2.png}}
  \subfloat[2x2+MINERvA] {\includegraphics[width=0.45\textwidth]{plots/2x2_minerva_plots/H_cont_eff_2x2_MINERvA.png}}
  \caption{Efficiency for containing hadronic showers, in the 2x2-only, and 2x2+MINERvA, as a function of hadronic shower energy and angle w.r.t the incoming neutrino direction. Containment is defined as $\geq$90\% of the energy being deposited in an active volume of a detector.}
  \label{fig:hadronic_containment}
\end{figure}
In Figure~\ref{fig:hadronic_containment}, the containment of hadron-induced showers is shown as a function of the true energy of the shower, and its angle w.r.t the incoming neutrino beam direction. Showers are defined as being contained when $\geq$90\% of the true energy of the shower is deposited inside the active 2x2 volume, or the MINERvA component if applicable. As expected, including a MINERvA component downstream of the 2x2 module increases the efficiency for angles $\theta \lesssim 30^{\circ}$, which dramatically increases the containment of high energy $E \gtrsim 0.5$ GeV hadronic showers, which tend to be forward-going.

\begin{figure}[htb]
  \centering
  \subfloat[2x2-only]    {\includegraphics[width=0.45\textwidth]{plots/2x2_minerva_plots/Pi0_cont_eff_2x2.png}}
  \subfloat[2x2+MINERvA] {\includegraphics[width=0.45\textwidth]{plots/2x2_minerva_plots/Pi0_cont_eff_2x2_MINERvA.png}}
  \caption{Efficiency for containing both photon-induced showers from $\pi^{0}$ decays, in the 2x2-only, and 2x2+MINERvA, as a function of the $\pi^{0}$ kinetic energy and angle w.r.t the incoming neutrino direction. Containment is defined as $\geq$90\% of the energy being deposited in an active volume of a detector.}
  \label{fig:pi0_containment}
\end{figure}
As discussed previously in this note, as the 2x2 module will not be placed in a test beam prior to installation in the NuMI beam at Fermilab, measurements in which the energy scale of the 2x2 can be calibration will be vital to assess the quality of energy reconstruction in the detector. The containment of both photons from a $\pi^{0}$ decay provides an appropriate in situ measurment of the energy reconstruction capabilities. In Figure~\ref{fig:pi0_containment}, the efficiency to contain 90\% of the energy from both photon-induced showers from a $\pi^{0}$ decay within the active volume of the 2x2, or the MINERvA component if relevant, is shown as a function of the $\pi^{0}$ kinetic energy and angle w.r.t the incoming neutrino beam. There is a significant increase in efficiency for all kinetic energies above a few hundred MeV, particularly for high energy ($E_{\pi^{0}} \gtrsim 1$ GeV) pions, which are produced in the forward direction. Although the dead space between the ArconCube 2x2 active volume and the MINERvA component complicates this picture somewhat, it is clear that including a large portion of MINERvA would give much greater statistics for this benchmark test of the ArgonCube detector performance.

\subsubsection{Neutron tagging studies}

\begin{figure}[htb]
  \centering
  \subfloat[2x2]    {\includegraphics[width=0.45\textwidth]{plots/2x2_minerva_plots/recoils_vs_E_proton_2x2.png}}
  \subfloat[MINERvA] {\includegraphics[width=0.45\textwidth]{plots/2x2_minerva_plots/recoils_vs_E_proton_MINERvA.png}}
  \caption{Number of neutron-induced proton recoils as a function of proton energy, which originate from an interaction vertex in the 2x2 active volume, seen in both the 2x2 ative volume, and the downstream MINERvA detector.}
  \label{fig:neutron_tag_minerva}
\end{figure}

As discussed in Section~\ref{sec:2x2_neutron}, one key detector physics goal with ProtoDUNE-ND is to determine whether neutron-induced proton recoils can be identified in a LAr TPC, specifically in ArgonCube. The ability to identify and measure neutrons produced in neutrino interactions is of great interest to DUNE.  At the far detector recoil protons can be identified and easily associated to the neutrino interaction.  However, at the near detector, confusion due to multiple neutrino interactions in the same beam spill poses a unique challenge.  Because neutrons can travel $\mathcal{O}\left(1\right)\,\mathrm{m}$ in LAr without interacting, and proton recoils from fast neutrons typically deposit energy on a single pixel and thus contain no directionality, event association is not possible without matching the charge deposit to an ArCLight optical flash with fast timing resolution.
 
Additionally, it may be possible to measure the neutron energy from time of flight in the DUNE ND using the ECAL with very fast, sub-nanosecond timing resolution. This will require matching muon tracks from either the LAr or HPGAR TPCs to hits in the ECAL to reconstruct the neutrino interaction vertex time with high precision, and also identify and timestamp a subsequent neutron interaction in the scintillator tiles of the ECAL. This would give the DUNE ND unprecedented ability to make measurements of the neutron energy spectrum in neutrino-argon interactions. This technique has not been tested in a high rate environment. Because the neutrons may propagate for $\mathcal{O}\left(10\right)\,\mathrm{ns}$, even a very fast detector may suffer from confusion due to pile-up.

MINERvA can detect neutron-induced proton recoils down to energies of a few MeV, and measure the 3D position of a recoil with a threshold of 20~MeV.  MINERvA has an established neutron reconstruction and a relatively well-understood detector response.  As shown in Fig. 24, it will be possible to reconstruct neutrons originating in the LAr of the ArgonCube 2x2 Demonstrator by their interactions in MINERvA. The ability to match both muons and neutrons originating in LAr to a fast-timing scintillator detector would be a direct test of the feasibility of this technique in DUNE ND. This has profound impact on the design of the ECAL, which would need to be optimized for both EM and neutron reconstruction if this technique is demonstrated to be viable. 
 
 
\FloatBarrier

\section{Outlook and timeline}
\label{sec:outlook}

\todo{Add dates for FNAL local infrastructure to be completed}\\
\todo{Add relevant dates for HPTPC/3DST if relevant, or branch points for decisions to be made if they cannot be ready in time}\\
\todo{Are there relevant LArPix dates to be included here? -- Dan}
\begin{itemize}
\item {\bf December 2018:} complete remaining R\&D tests to be carried out before the 2x2 is built. Two outstanding tests planned:
  \begin{enumerate}
  \item Although initial tests with a resistive field shell were successful with a small TPC~\cite{argoncube_fd}, a test with a larger TPC with a resistive field shell will be carried out at the University of Bern by the end of 2018. The primary purpose will be to conduct a studies of breakdown studies, to investigate the power dissipation and whether the field shell is damaged in the event of a breakdown, as well as to check the field uniformity and stability. If for whatever reason a problem is found with the resistive field shell design for the 2x2, the backup option will be to revert to using a field cage, as in Ref.~\cite{argoncube_loi}.
  \item Tests of the cryogenics and the purity module will be performed by the end of 2018. Extraction and reinsertion of a test module with the argon recirculation system will be performed. This will check that the argon purification system works as expected for the ArgonCube 2x2 Demonstrator module, and will help inform the final design choice of the full-size ArgonCube, for which different potential purification systems are being considered (see Section~\ref{sec:2x2-design}).
  \end{enumerate}
\item {\bf January 2019:} first module for the 2x2 complete, ready for first local ArgonCube 2x2 Demonstrator tests.
\item {\bf June 2019:} all four modules completed.
\item {\bf June--August 2019:} tests of the complete ArgonCube 2x2 Demonstrator to be performed at the University of Bern. Verification that the slow control works as expected. Basic DAQ tests, and cosmic muon sample collected.
\item {\bf December 2019:} transport the ArgonCube 2x2 Demonstrator to FNAL
\item {\bf January--February 2019:} detector commissioning at FNAL.
\item {\bf March 2019:} ready to take data.
\item \todo{This timeline is obviously very tentative, but we should definitely check what the NuMI run plan is... there's not much point pushing an ambitious schedule if we arrive in a long shutdown}
\end{itemize}

\section{Conclusions}
\label{sec:conclusions}

In this work, we have outlined the detector physics potential of the ProtoDUNE-ND testbench experiment in the MINOS-ND hall at Fermilab. At the heart of ProtoDUNE-ND sits the ArgonCube 2x2 demonstrator module, which is currently being commissioned, and will be moved to Fermilab by early 2020. The set-up of ProtoDUNE-ND is intended to be flexible, to allow for new modules testing different aspects of the future DUNE-ND design to be installed at different times. This is facilitated by the moveable cryogenic support systems which need to be tested for the DUNE-PRISM baseline design concept, and which will allow the ArgonCube 2x2 to be moved, and the ProtoDUNE-ND arrangement to be reconfigured as a result. The detector physics studies outlined in this document will provide vital inputs to the full-scale DUNE-ND design, and provide a much-needed intermediate scale test, scaling up the small R\&D tests which have already been carried out, and allowing for long-term stability tests to be carried out.

In this document, we have also outlined the detector physics case for incorporating elements of the soon to be decommissioned MINERvA detector into the ProtoDUNE-ND setup. All DUNE-ND designs considered in Ref.~\ref{dune_ndcsg} include some fast scintillator detector, but no prototype is foreseen at this time for ProtoDUNE-ND. MINERvA components would fill that gap, and provide an essential test of combined reconstruction across different detectors will very different readout technology and times. Additionally, we briefly discussed the possibility to include the MINOS-ND into ProtoDUNE-ND, which is due to be decommissioned along with the MINERA experiment in mid-2019. In combination, using elements of MINERvA and MINOS-ND in ProtoDUNE-ND would allow us to contain all particles in a large fraction of events, and provide a realistic test of the energy recontruction capabilities of the final DUNE-ND. This would be invaluable for DUNE-ND design studies, and would greatly enhance the impact of the ProtoDUNE-ND test.


%\begin{acknowledgments}
%Thanks for all the \$\$\$
%\end{acknowledgments}

\bibliography{bibliography.bib}% Produces the bibliography via BibTeX.

%\appendix
%\input{an_appendix}

\end{document}
