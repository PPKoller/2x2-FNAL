\section{Incorporating other detector prototypes into ProtoDUNE-ND}
\label{sec:MINERvA_MINOS}

\begin{figure}[htb]
	\centering
	\includegraphics[width=0.8\textwidth]{{plots/EventDisplays/8.17GeV_rectangle_crop}.png}
	\caption{Example ArgonBox simulated event for an 8.17 GeV $\nu_{\mu}$--argon neutral-current multi-pion interaction, in which the pions are not contained in the module. Energy deposits in a bulk volume of LAr are color-coded according to the particle type: $\pi^{\pm}$ --- blue; $\mu^{\pm}$ --- purple; $e^{+}$ --- green; $e^{-}$ --- yellow; proton --- red; recoiling nuclei --- black. The event vertex was randomly placed inside the active volume of the ArgonCube 2x2 Demonstrator, the geometry for which is superimposed on these images, but which is not simulated by ArgonBox.}
	\label{fig:leaky_event}
\end{figure}
Given the ArgonCube 2x2 Demonstrator's size and the relatively high energy NuMI ME beam, many events will not be contained in the 2x2 alone. Figure~\ref{fig:leaky_event} shows a neutral current event where many pions are produced, but in which the pions and subsequent hadronic showers extend far beyond the detector. Indeed, many such events would not be contained in the full ArgoNCube deployment at the DUNE ND, in the LBNF beamline, and for charged-current events, the muon will be uncontained most of the time. For this reason, and because it is not possible to magnetize the large LAr component, further, magnetized, tracking detectors are proposed downstream of ArgonCube in the full DUNE ND complex, and the requirement to tag side-escaping particles is discussed in Ref.~\cite{dune_ndcsg}. Broadly speaking, three additional subdetector options are considered for the DUNE-ND design, in addition to ArgonCube~\cite{dune_ndcsg}, as introduced in Section~\ref{sec:tracking_detectors}: a scintillator tracking detector; a magnetized low-density tracking detector; and a Cosmic Ray Tagging (CRT) system.

Prototypes for these detectors are also being planned, and the ProtoDUNE-ND complex is intended to evolve over time to accommodate them. As has been remarked, the cryogenic system for the ArgonCube 2x2 will be moveable to test key components of the DUNE-PRISM technical design, which will allow ProtoDUNE-ND to be easily reconfigured to accommodate any future prototype detectors. In this section, we discuss how adding those additional detectors will enhance the neutrino engineering studies possible with the ArgonCube 2x2 Demonstrator alone. With multiple subdetectors included in ProtoDUNE-ND, multi-detector reconstruction capabilities can be developed and tested. Additionally, if the sign and momentum of escaping hadrons and muons can be measured, it may be possible to make physics measurements with ProtoDUNE-ND, which would be beneficial to the overall DUNE program.

\subsection{Scintillator tracking detector}
\label{sec:minerva}
Two distinct scintillator options are currently being considered for the DUNE ND design, as introduced in Section~\ref{sec:tracking_detectors}. A large, fully active, Three-Dimensional Scintillator Tracker (3DST), and scintillator components downstream of the HPgTPC to tag escaping neutral particles. %As discussed in Section~\ref{sec:detector-physics-studies}, a large fraction of particles will not be contained in the 2x2 alone. A downstream scintillator detector would recover many of these particles, and would make it possible to benchmark the 2x2 response to a wider range of particle energies, and therefore the expected DUNE phase-space.

To investigate the effect that adding a scintillator component, and generically a downstream tracking detector, to ProtoDUNE-ND, a simulation was performed where a generic scintillator box was included downstream of the ArgonCube 2x2 Demonstrator. Neutrino interactions are generated in the ArgonCube active volume, and propagated through an approximation of the ArgonCube 2x2 demonstrator and scintillator block.  The scintillator detector used is simply a rectangular box, \SI[product-units=repeat]{1.4x1.4}{\metre\squared} in the dimensions transverse to the beam, making it large enough to cover the downstream face of the 2x2 active volume. In the beam direction, the scintillator is split into 

, and~\todo{how long?}  The simulation includes the most downstream 12 modules (24 planes) of the tracker region, as well as the full downstream ECAL and HCAL regions.  The rectangular box represents the central part of the MINERvA inner detector, and is large enough to cover the entire ArgonCube active volume. An example event is shown in Figure~\ref{fig:2x2+MINERvA_event}, and can be compared with 2x2 only events in Figures~\ref{fig:argonbox_event_display} and~\ref{fig:leaky_event}. Note that this simulation only included the ArgonCube cryostat and MINERvA detector components, no material was included outside these (so escaping particles simply leave without ever re-interacting). This is unlike the previously described ArgonBox simulation, where a large box of argon was simulated (so escaping particles still re-interact). Events were again distributed uniformly throughout the ArgonCube active volume.
\begin{figure}[htb]
  \centering
  \includegraphics[width=0.8\textwidth]{{plots/Event_Displays_2x2_MINERvA/MINERvA_full_e70_rectangle_crop}.png}
  \caption{Example simulated event for a 7.0 GeV $\nu_{\mu}$--argon charged-current interaction, in which particles not contained in the ArgonCube 2x2 enter the scintillator block detector. Energy deposits are color-coded according to the particle type: $\pi^{\pm}$ --- blue; $\mu^{\pm}$ --- purple; $e^{+}$ --- green; $e^{-}$ --- yellow; proton --- red; recoiling nuclei --- black. The event vertex was randomly placed inside the active volume of the 2x2 Demonstrator module.}
  \label{fig:2x2+MINERvA_event}
\end{figure}

In the following, we discuss potential detector physics studies, or improvements to detector physics studies previously discussed in the ArgonCube 2x2-only case (in Section~\ref{sec:detector-physics-studies}), incorporating elements of the MINERvA detector downstream of the ArgonCube 2x2 module.

\subsubsection{Track matching}
All DUNE ND designs considered in Ref.~\cite{dune_ndcsg} include some fast scintillator component, downstream of the LAr ArgonCube component, and downstream of a low-density GAr TPC tracker, to tag escaping particles, photons, and possibly neutrons. There is a significant reconstruction challenge in matching the escaping tracks from the LAr component, with the signals in the scintillator, given the slow charge readout in the LAr TPC, and the high multiplicity DUNE-ND environment.

\begin{figure}[htb]
  \centering
  \includegraphics[width=0.6\textwidth]{plots/2x2_minerva_plots/track_mathch_multiplicity.png}
  \caption{Simulated number of true tracks produced by simulated GENIE interactions in the ArgonCube 2x2 active volume, which deposit energy in both the 2x2 module, and the MINERvA component positioned downstream of the 2x2.}
  \label{fig:track_multiplicity_min}
\end{figure}
Many tracks produced in the LAr volume are not contained by the ArgonCube 2x2 module, and the majority will escape downstream. In Figure~\ref{fig:hadronic_containment}, the multiplicity of tracks which deposit charge in both the ArgonCube 2x2 module and the MINERvA component, included in the simulation described above, are shown. Full DUNE-ND events are likely to have an even higher LAr to scintillator track multiplicity due to the pile-up in the much larger 35 t ArgonCube LAr detector. But it is clear from Figure~\ref{fig:track_multiplicity_min} that including MINERvA elements in the ProtoDUNE-ND tests would provide useful data with which to start tackling this reconstruction problem.

\begin{figure}[htb]
  \centering
  \includegraphics[width=0.6\textwidth]{plots/2x2_minerva_plots/track_mathch_topo.png}
  \caption{Simulated number of true tracks produced by simulated GENIE interactions in the ArgonCube 2x2 active volume, which exit the downstream face of the 2x2 module, relative to the number of tracks which enter the upstream face of the downstream MINERvA component, event by event.}
  \label{fig:track_multiplicity_topo}
\end{figure}
As can be seen from the event display shown in Figure~\ref{fig:2x2+MINERvA_event}, events in which tracks escaping the 2x2 active volume may re-interact in the surrounding LAr bath before entering the MINERvA component included in this simulation, thus making the event more confusing, and difficult to assess reconstruction performance with. Figure~\ref{fig:track_multiplicity_topo} shows the multiplicity of tracks exiting the downstream face of the 2x2 active volume downstream, compared with the number of tracks entering the upstream face of the MINERvA component included in the simulation. The distribution is fairly diagonal, suggesting that although complicated event topologies exist, the events will not be too confused to use for these studies. Note also that this problem could be dramatically reduced by partially instrumenting the dead region between the two detectors.

\subsubsection{Acceptance studies}
\label{sec:minerva-acceptance}
The inclusion of MINERvA in ProtoDUNE-ND will improve the acceptance of particles for various studies. Here, we show how the efficiency for contained events compares for the 2x2+MINERvA setup desribed above, with MINERvA components located downstream of the ArgonCube 2x2 Demonstrator module, and for the 2x2-only case.

\begin{figure}[htb]
  \centering
  \subfloat[2x2-only]    {\includegraphics[width=0.45\textwidth]{plots/2x2_minerva_plots/H_cont_eff_2x2.png}}
  \subfloat[2x2+MINERvA] {\includegraphics[width=0.45\textwidth]{plots/2x2_minerva_plots/H_cont_eff_2x2_MINERvA.png}}
  \caption{Efficiency for containing hadronic showers, in the 2x2-only, and 2x2+MINERvA, as a function of hadronic shower energy and angle w.r.t the incoming neutrino direction. Containment is defined as $\geq$90\% of the energy being deposited in an active volume of a detector.}
  \label{fig:hadronic_containment}
\end{figure}
In Figure~\ref{fig:hadronic_containment}, the containment of hadron-induced showers is shown as a function of the true energy of the shower, and its angle w.r.t the incoming neutrino beam direction. Showers are defined as being contained when $\geq$90\% of the true energy of the shower is deposited inside the active 2x2 volume, or the MINERvA component if applicable. As expected, including a MINERvA component downstream of the 2x2 module increases the efficiency for angles $\theta \lesssim 30^{\circ}$, which dramatically increases the containment of high energy $E \gtrsim 0.5$ GeV hadronic showers, which tend to be forward-going.

\begin{figure}[htb]
  \centering
  \subfloat[2x2-only]    {\includegraphics[width=0.45\textwidth]{plots/2x2_minerva_plots/Pi0_cont_eff_2x2.png}}
  \subfloat[2x2+MINERvA] {\includegraphics[width=0.45\textwidth]{plots/2x2_minerva_plots/Pi0_cont_eff_2x2_MINERvA.png}}
  \caption{Efficiency for containing both photon-induced showers from $\pi^{0}$ decays, in the 2x2-only, and 2x2+MINERvA, as a function of the $\pi^{0}$ kinetic energy and angle w.r.t the incoming neutrino direction. Containment is defined as $\geq$90\% of the energy being deposited in an active volume of a detector.}
  \label{fig:pi0_containment}
\end{figure}
As discussed previously in this note, as the 2x2 module will not be placed in a test beam prior to installation in the NuMI beam at Fermilab, measurements in which the energy scale of the 2x2 can be calibration will be vital to assess the quality of energy reconstruction in the detector. The containment of both photons from a $\pi^{0}$ decay provides an appropriate in situ measurment of the energy reconstruction capabilities. In Figure~\ref{fig:pi0_containment}, the efficiency to contain 90\% of the energy from both photon-induced showers from a $\pi^{0}$ decay within the active volume of the 2x2, or the MINERvA component if relevant, is shown as a function of the $\pi^{0}$ kinetic energy and angle w.r.t the incoming neutrino beam. There is a significant increase in efficiency for all kinetic energies above a few hundred MeV, particularly for high energy ($E_{\pi^{0}} \gtrsim 1$ GeV) pions, which are produced in the forward direction. Although the dead space between the ArconCube 2x2 active volume and the MINERvA component complicates this picture somewhat, it is clear that including a large portion of MINERvA would give much greater statistics for this benchmark test of the ArgonCube detector performance.

\subsubsection{Neutron tagging studies}

\begin{figure}[htb]
  \centering
  \subfloat[2x2]    {\includegraphics[width=0.45\textwidth]{plots/2x2_minerva_plots/recoils_vs_E_proton_2x2.png}}
  \subfloat[MINERvA] {\includegraphics[width=0.45\textwidth]{plots/2x2_minerva_plots/recoils_vs_E_proton_MINERvA.png}}
  \caption{Number of neutron-induced proton recoils as a function of proton energy, which originate from an interaction vertex in the 2x2 active volume, seen in both the 2x2 ative volume, and the downstream MINERvA detector.}
  \label{fig:neutron_tag_minerva}
\end{figure}

As discussed in Section~\ref{sec:2x2_neutron}, one key detector physics goal with ProtoDUNE-ND is to determine whether neutron-induced proton recoils can be identified in a LAr TPC, specifically in ArgonCube. The ability to identify and measure neutrons produced in neutrino interactions is of great interest to DUNE.  At the far detector recoil protons can be identified and easily associated to the neutrino interaction.  However, at the near detector, confusion due to multiple neutrino interactions in the same beam spill poses a unique challenge.  Because neutrons can travel $\mathcal{O}\left(1\right)\,\mathrm{m}$ in LAr without interacting, and proton recoils from fast neutrons typically deposit energy on a single pixel and thus contain no directionality, event association is not possible without matching the charge deposit to an ArCLight optical flash with fast timing resolution.
 
Additionally, it may be possible to measure the neutron energy from time of flight in the DUNE ND using the ECAL with very fast, sub-nanosecond timing resolution. This will require matching muon tracks from either the LAr or HPGAR TPCs to hits in the ECAL to reconstruct the neutrino interaction vertex time with high precision, and also identify and timestamp a subsequent neutron interaction in the scintillator tiles of the ECAL. This would give the DUNE ND unprecedented ability to make measurements of the neutron energy spectrum in neutrino-argon interactions. This technique has not been tested in a high rate environment. Because the neutrons may propagate for $\mathcal{O}\left(10\right)\,\mathrm{ns}$, even a very fast detector may suffer from confusion due to pile-up.

MINERvA can detect neutron-induced proton recoils down to energies of a few MeV, and measure the 3D position of a recoil with a threshold of 20~MeV.  MINERvA has an established neutron reconstruction and a relatively well-understood detector response.  As shown in Fig. 24, it will be possible to reconstruct neutrons originating in the LAr of the ArgonCube 2x2 Demonstrator by their interactions in MINERvA. The ability to match both muons and neutrons originating in LAr to a fast-timing scintillator detector would be a direct test of the feasibility of this technique in DUNE ND. This has profound impact on the design of the ECAL, which would need to be optimized for both EM and neutron reconstruction if this technique is demonstrated to be viable. 
 
 
\FloatBarrier
\subsection{Magnetized low-density tracking detector}
The DUNE ND will employ a downstream tracking detector to measure forward, high-energy muons, and will be sufficiently wide as to contain muons at high angles.  However, for wide-angle muons, tracks will only be contained when the vertex is far from the edges, and will often exit the TPC and not be reconstructed.  This sample could be recovered if the muon momentum can be measured reliably from multiple coulomb scattering (MCS). The far detector will also use MCS for muons which exit the detector.

Such measurements have been carried out in a LAr TPC by MicroBooNE~\cite{Abratenko:2017nki}, although the momentum range is somewhat limited as the only validation sample available is composed of muons which stop in the detector, for which a momentum by range measurement can be made. However, such measurements may be more challenging in a modular TPC such as ArgonCube, where tracks are necessarily broken into segments which are read out separately, and then combined in downstream software.

Although there are only four modules in the ArgonCube 2x2 Demonstrator, each is split into two TPCs, so tracks which exit the downstream face of the 2x2 module may be sampled by up to four independent charge readout planes. With a downstream detector capable of making a precise muon momentum measurement, it would be possible to demonstrate that multiple coulomb scattering will be a viable technique for making momentum measurements in the full ArgonCube deployment in the DUNE ND. Given the hard muon momentum spectrum expected in the NuMI on-axis beam (shown in Figure~\ref{fig:momenta}), it would be hard to make a reliable momentum measurement with a small demonstrator tracker module, or indeed with the MINERvA components proposed in Section~\ref{sec:minerva}, as few would be contained. Such measurements could be made, in the MINOS-ND, which is capable of making momentum measurements for all relevant muon momenta by curvature, or by range.


\section{Cosmic Ray Tagger}

In practice, these panels could be used to better identify a sample of fully contained neutrino events for the studies outlined in Sections~\ref{sec:minerva-acceptance}. Additionally, these cosmic ray tagging modules could then be used to validate space charge build-up and electric field uniformity studies outlined in Section~\ref{sec:efield}, and the cosmic suppression studies described in Section~\ref{sec:cosmic-suppression}, for which an additional scintillator panel would be required.

\subsection{Space charge and electric field uniformity}
\label{sec:efield}
A concern for LAr detectors in a high intensity beam is the build up of space charge --- long-lived argon ions which drift slowly towards the cathode --- and possible affects on the uniformity of the electric field which may accumulate over time. In currently operating and near-future LAr detectors~\cite{Ereditato:2014lra, Antonello:2015lea}, both cosmic tracks and UV lasers are used to calibrate for distortions in the electric field. Both the UV laser track and high energy cosmic muons are expected to leave straight tracks in the detector. If the drift field is not uniform across the detector, ionization electrons produced along the length of this track will not drift at the same speed, or with a constant direction, and will result in a distorted track at the readout plane. By comparing the reconstructed and expected track, a map of the electric field distortion can be built up for calibration purposes.

Assuming that the ArgonCube 2x2 Demonstrator is equipped with scintillator panels to tag cosmic tracks using a timing coincidence between two sides of the detector, and reasonable spatial resolution on those scintillator paddles, electric field distortions could be measured in the 2x2 Demonstrator module. By looking at beam-on, and beam-off data, it would be possible to look at the possible affect of space charge build-up over time due to the high event rate in the NuMI beam. If significant space charge build-up were observed, this would inform the future ArgonCube design for the DUNE ND, as a higher drift field strength would be required.

Additionally, although the electric-field uniformity of the resistive field shell will be checked in a small scale LAr TPC at Bern, a check of the electric-field uniformity and stability over time for full-size ArgonCube modules would be a valuable final validation of the design.

\subsection{Cosmic and Rock Muon suppression}
\label{sec:cosmic-suppression}

The DUNE ND is located underground with a \SI{50}{\metre} overburden, that reduces the cosmic flux by orders of magnitude compared to surface detectors. The cosmic rate in MINERvA (currently in the MINOS ND hall where ProtoDUNE-ND will be sited) is \SI{18}{\hertz}, for $\sim$\SI{6}{\metre\squared}. For the ArgonCube 2x2, in the $\sim$\SI{150}{\micro\second} readout window, assuming $\sim$\SI{2}{\metre\squared}, the rate of cosmics muons is 0.001 per readout window. Reading out around spills only, will result in a coincident cosmic muon once every 20 minutes.
In DUNE ND we expect $\sim$10 rock muons per spill, which will be on average $\sim$\SI{1}{\micro\second} apart in time. They are unlikely to overlap any neutrino activity in 3D, but the fast timing will provide an additional handle.
A key design choice for the DUNE ND is whether or not a muon tagging system is required --- for example, a series of scintillator planes surrounding the LAr TPC, as for SBND and MicroBooNE~\cite{CRT}. The proposed ProtoDUNE-ND test experiment can help inform this design choice, assuming that the ArgonCube 2x2 Demonstrator module is equipped with scintillator paddles.

By using the scintillator paddles to tag cosmic or rock events using a timing coincidence, methods for rejecting them can be validated independently. For example, we expect that the good timing resolution and light localization from ArcLight will allow cosmics out of the beam window to be rejected with a high efficiency. Additionally, if a cosmic muon traverses a pixel plane, or an ArcLight plane, we would expect to see a large charge deposition or a large number of photoelectrons measured, which could be an additional way to reject cosmic muons. Both of these methods can be validated with ProtoDUNE-ND.
