\section{Conclusions}
\label{sec:conclusions}

In this work, we have outlined the detector physics potential of the ProtoDUNE-ND test-bench experiment in the MINOS-ND hall at Fermilab, which is intended to be a ``neutrino engineering'' test. At the heart of ProtoDUNE-ND sits the ArgonCube 2x2 Demonstrator module, which is currently being commissioned, and will be moved to Fermilab by early 2020. The set-up of ProtoDUNE-ND is intended to be flexible, to allow for new modules testing different aspects of the future DUNE-ND design to be installed at different times. This is facilitated by the moveable cryogenic support systems which need to be tested for the DUNE-PRISM baseline design concept, and which will allow the ArgonCube 2x2 to be moved, and the ProtoDUNE-ND arrangement to be reconfigured as a result. The detector physics studies outlined in this document will provide vital inputs to the full-scale DUNE-ND design, and provide a much-needed intermediate scale test, scaling up the small R\&D tests which have already been carried out, and allowing for long-term stability tests to be carried out.

In this document, we have also outlined the potential for additional detector physics studies possible if prototypes of other potential DUNE ND subdetector modules become available. These studies would help inform key outstanding DUNE ND design questions as discussed in Ref.~\cite{dune_ndcsg}. As the cryogenics for the ArgonCube 2x2 are designed to be moveable in order to test key components of DUNE-PRISM's technical design, it will be possible to reconfigure ProtoDUNE-ND in order to accommodate any future prototypes for these additional subdetector modules as they become available.
