\section{Conclusions}
\label{sec:conclusions}

In this work, we have outlined the detector physics potential of the ProtoDUNE-ND test-bench experiment in the MINOS-ND hall at Fermilab, which is intended to be a ``neutrino engineering'' test. At the heart of ProtoDUNE-ND sits the ArgonCube 2x2 Demonstrator module, which is currently being commissioned, and will be moved to Fermilab by early 2020. The set-up of ProtoDUNE-ND is intended to be flexible, to allow for new modules testing different aspects of the future DUNE-ND design to be installed at different times. This is facilitated by the moveable cryogenic support systems which need to be tested for the DUNE-PRISM baseline design concept, and which will allow the ArgonCube 2x2 to be moved, and the ProtoDUNE-ND arrangement to be reconfigured as a result. The detector physics studies outlined in this document will provide vital inputs to the full-scale DUNE-ND design, and provide a much-needed intermediate scale test, scaling up the small R\&D tests which have already been carried out, and allowing for long-term stability tests to be carried out.

In this document, we have also outlined the detector physics case for incorporating a downstream tracker to ProtoDUNE-ND. 
This would improve containment and provide an essential test of combined reconstruction across different detectors with very different readout technology and times. 
All DUNE-ND designs considered in Ref.~\cite{dune_ndcsg} include some fast scintillator detector, but no prototype is foreseen at this time for ProtoDUNE-ND. 
As an illustrative example, we have considered incorporating elements of the soon to be decommissioned MINERvA detector into the ProtoDUNE-ND setup, provided all infrastructure and resource availability needs can be met. 
Additionally, we briefly discussed the possibility to include the MINOS-ND into ProtoDUNE-ND, which is due to be decommissioned along with the MINERA experiment in mid-2019. In combination, using elements of MINERvA and MINOS-ND in ProtoDUNE-ND would allow us to contain all particles in a large fraction of events, and provide a realistic test of the energy reconstruction capabilities of the final DUNE-ND. This would be invaluable for DUNE-ND design studies, and would enhance the impact of the ProtoDUNE-ND test.
