\section{Introduction}
\label{sec:introduction}

DUNE is a highly ambitious next-generation accelerator neutrino oscillation experiment~\cite{DUNE, DUNE2}. DUNE will utilise an intense beam of predominantly muon (anti)neutrinos produced at Fermilab, will sample the unoscillated beam on the Fermilab site, 574m from the proton target, and then will sample the oscillated beam 1300km away at the Sanford Underground Research Facility (SURF) in South Dakota, using four 10kt liquid argon (LAr) target modules. By using LAr time projection chambers (TPCs), DUNE benefits from the exquisite precision to look at the particles escaping the nucleus, and measuring the neutrino energies with unparalleled precision. An extensive research and development (R\&D) program is currently ongoing to deliver detectors capable of fulfilling DUNE's physics goals. The unprecedented size of the DUNE far detectors~\cite{DUNE_IDR_v1, DUNE_IDR_v2, DUNE_IDR_v3} has motivated large-scale LAr TPC tests, in the single- and dual-phase ProtoDUNE modules in CERN~\cite{Abi:2017aow, Agostino:2014qoa}, in themselves, the largest LAr TPCs built to date.

As the DUNE far detectors have LAr targets, it is highly desirable to also have a major LAr component for the DUNE near detector suite, in order to minimize cross section and detector systematic uncertainties for oscillation analyses~\cite{DUNE, DUNE2}. However, the intense neutrino flux and high event rate at the near detector makes traditional, monolithic, projective wire readout TPCs unsuitable, which has motivated a program of R\&D into a new LAr TPC approach, suitable for such a high rate environment, known as ArgonCube~\cite{argoncube_loi}. ArgonCube utilizes detector modularization to improve drift field stability, reducing high voltage (HV) and the LAr purity requirements; it uses a pixelized charge readout~\cite{pixels, larpix}, which provides unambiguous 3D imaging of particle interactions that drastically simplifying the reconstruction; and new dielectric light detection techniques with ArCLight~\cite{arclight}, which can be placed inside the field cage to increase light yield, and localization of light signals. Additionally, ArgonCube uses a resistive field shell, instead of traditional field shaping rings, to minimize the dead material and maximize active volume, and to minimize the power release in the event of a breakdown~\cite{argoncube_fd}. Such an optically segmented, pixel readout LAr TPC has been recommended as the major LAr component for the DUNE near detector by the DUNE Near Detector Concept Study Group~\cite{dune_ndcsg}.

Early investigations into long drift lengths and high electric fields with the ArgonTube detector~\cite{argontube_design} found a number of technical issues and a higher than expected risk of breakdowns~\cite{argontube}, as well as testing cryogenic charge readout electronics~\cite{art_cold_ero}. Smaller, modular LAr TPCs were introduced as a robust solution to the high voltage problem, which reduces the LAr purity requirements due to the short drift lengths. The feasibility of pixelated readout was demonstrated using a 60-cm-drift pixel TPC located at the University of Bern in 2016~\cite{pixels}. LArIAT~\cite{lariat} has since operated a scaled-up version of this readout technique (renamed PixLAr) at Fermilab in 2017. Recent developments to the LArPix pixel readout electronics made at LBNL have removed the ambiguities present for the earlier tests~\cite{larpix}. The 60-cm-drift pixel TPC, and a 10-cm version at LBNL, were used to demonstrate cold amplification and cold digitization with the updated LArPix electronics~\cite{larpix}. A field-shell demonstrator TPC was tested successfully at the University of Bern in 2018~\cite{argoncube_fd}. Tests of the ArCLight dielectric photon detector have also been carried out in LAr, as described in Ref.~\cite{arclight}. With the various technological developments demonstrated with small-scale TPCs, the next necessary step in the ArgonCube program is to demonstrate the scalability of the pixelized charge readout and light detection systems, and to show that information from separate modules can be combined to produce high quality event reconstruction for particle interactions. To that end, a mid-scale (1.4~m $\times$ 1.4~m $\times$ 1.2~m) modular TPC, dubbed the ArgonCube 2x2 Demonstrator, with 4 independent LAr TPC modules arranged in a 2x2 grid has been designed, and is currently under construction at the University of Bern. Given the environment that the DUNE near detector will operate in, it is desirable to operate the ArgonCube 2x2 Demonstrator in an intense, few-GeV, neutrino beam.

In this document, we propose moving the ArgonCube 2x2 Demonstrator into the NuMI on-axis neutrino beamline at FNAL, to serve as the core of a prototype for the near detector, or ProtoDUNE-ND. A detailed description of the proposed location and the ArgonCube 2x2 Demonstrator module can be found in Section~\ref{sec:protodune-nd}. A description of the detector physics studies to be performed in Section~\ref{sec:detector-physics-studies}. The outlook for the test modules and a timeline in Section~\ref{sec:outlook}. Conclusions are presented in Section~\ref{sec:conclusions}.\\\\

\todo{The last paragraph will need to be reworked to include the HPgTPC if possible. If not, we should discuss the proximity to MINERvA/MINOS, and the possiblity to benefit from them.}
