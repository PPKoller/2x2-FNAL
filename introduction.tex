\section{Introduction}
\label{sec:introduction}

DUNE is a highly ambitious next-generation accelerator neutrino oscillation experiment~\addcite. DUNE will utilise an intense beam of predominantely muon (anti)neutrinos produced at Fermilab, sample the unoscillated beam on the Fermilab site, ~650m \todo{check!} from the proton target, and then sample the unoscillated beam XXkm away at XXX, using 4 10kt liquid argon (LAr) target modules. By using LAr time projection chambers (TPCs), DUNE benefits from the exquisite precision to look at the particles escaping the nucleus, and measuring the neutrino energies with unparalleled precision. \todo{Read a DUNE intro paper and refine this to toe the party line.} An extensive research and development (R\&D) program is currently ongoing to deliver detectors capable of fulfilling DUNE's physics goals. The unprecedented size of the DUNE far detectors has motivated large-scale LAr TPC tests, in the single- and dual-phase ProtoDUNE modules in CERN, in themselves, the largest LAr TPCs built to date. Similarly, the intense neutrino flux at the near detector has motivated a program of R\&D into pixelized charge readout, and various improvements with respect to traditional LAr TPCs, in the ArgonCube program \todo{make this less shit, and say modularized}. A necessary final step in this program is a full size test of the ArgonCube modules \todo{mention that ProtoDUNE are conceived as full-size modules of the FD} in a neutrino beam to check that it can perform as required by DUNE, and to guide reconstruction development work to be ready for the full DUNE near detector when it is built in XXXX.

Introduction to DUNE and the DUNE ND. Describe and reference past R\&D work done for ArgonCube, ArgonTube, etc. \\ \\

Add similar descriptions for HPTPC and 3DST if they will be able to form a part of this proposal. If not, I guess we should comment on the general multi-detector ND design, but not go into details. {\it If we ultimately do not include anything other than the 2x2, we should comment on the location of MINERvA/MINOS-ND in the hall, and the possibility to test reconstruction with multiple detectors... maybe that's a useful think to also test?}\\\\

Proposal on the table is to put test modules into the NuMI on-axis neutrino beamline at FNAL to act as ProtoDUNE-ND --- vital test of the near detectors before investing in the full DUNE ND. A description of the proposal location and test modules to be used in Section~\ref{sec:protodune-nd}. A description of the detector physics studies to be performed in Section~\ref{detector-physics-studies}. The outlook for the test modules and a timeline in Section~\ref{sec:outlook}. {\it Probably want to have a strong show of support from FNAL in this section --- e.g. if they promose concrete infrastructure support, it should go here, as well as in the description of the location for the test.} Conclusions in Section~\ref{sec:conclusions}.


